% Copyright 2007 by Till Tantau
%
% This file may be distributed and/or modified
%
% 1. under the LaTeX Project Public License and/or
% 2. under the GNU Public License.
%
% See the file doc/licenses/LICENSE for more details.



\documentclass{beamer}

%
% DO NOT USE THIS FILE AS A TEMPLATE FOR YOUR OWN TALKS�!!
%
% Use a file in the directory solutions instead.
% They are much better suited.
%


% Setup appearance:

\usetheme{Darmstadt}
%\usetheme{CambridgeUS}
%\usetheme{Berkeley}
%\usetheme{Hannover}
\usefonttheme[onlylarge]{structurebold}
\setbeamerfont*{frametitle}{size=\normalsize,series=\bfseries}
\setbeamertemplate{navigation symbols}{}
\setbeamercovered{transparent=30}
\setbeamertemplate{footline}{
    
    \leavevmode%
    \hbox{%
    \begin{beamercolorbox}[wd=.4\paperwidth,ht=2.25ex,dp=1ex,center]{author in head/foot}%
        \usebeamerfont{author in head/foot}\insertshortauthor
    \end{beamercolorbox}%

    \begin{beamercolorbox}[wd=.6\paperwidth,ht=2.25ex,dp=1ex,center]{title in head/foot}%
        \usebeamerfont{title in head/foot}\insertshorttitle\hspace*{3em}
        \insertframenumber{} / \inserttotalframenumber\hspace*{1ex}
    \end{beamercolorbox}}%
    \vskip0pt% 
    
    
}

% Standard packages

\usepackage[english]{babel}
\usepackage[latin1]{inputenc}
\usepackage{times}
\usepackage[T1]{fontenc}
\usepackage{graphicx}
%\usepackage{natbib}
\usepackage{enumerate}



% Setup TikZ

\usepackage{tikz}
\usetikzlibrary{arrows}
\tikzstyle{block}=[draw opacity=0.7,line width=1.4cm]


% Author, Title, etc.

\title[ 2014 fall updates ] 
{%
    Hardware Security Project
  %
}

\author[Boyou Zhou]
{
  \textit{Boyou Zhou\inst{*}}
}

\institute[Boston University, MA]
{
  \inst{*}
  Boston Univeristy, MA
}

\date[fall 2014]
{Created on Oct 4 2014, Modified on \today}



% The main document

\begin{document}

\begin{frame}
  \titlepage
\end{frame}

\begin{frame}{Outline}
  \tableofcontents
\end{frame}

\AtBeginSubsection{
    \frame<beamer>{ 
    \frametitle{Outline}   
    \tableofcontents[currentsection,currentsubsection]
    }
}


\section{Preparation Work}
\subsection{Contacting People}
\begin{frame}
    Oct 8 , 2014
\end{frame}


\begin{frame}{Three Parts of Work}
%    Picture Example
%    \begin{figure}[p]
%        \includegraphics[width=4in]{../doc_image/presentation/radio_freq.png}
%        \caption{Spectrum of electromagnetic waves}
%        \label{Spectrum of electromagnetic waves}
%    \end{figure}
    \begin{itemize}
        \item  \textit{Sheng Wei}
            \begin{itemize}
\pause
                \item   The main point of the paper is to develop an one-gate trojan trigger 
                circuit. The attack circuitry is on TrustHub. TrustHub has various kinds
                of attacking circtuits' verilog codes.
\pause
                \item   Sheng works on only a small part of the project for developing triggers.
                He can help us develop the trigger but not the attacking circtuit.
\pause
                \item   Leakage power analysis has been applied for detecting trojans that are 
                off at first and then turned on later. For example, trigger can comes from a
                counter.
            \end{itemize}
    \end{itemize}
\end{frame}

\begin{frame}{Second part}
    \begin{itemize}
        \item  \textit{Ronen}
            \begin{itemize}
\pause
                \item   The whole circuit's photonics response is simulated by adding up single gate 
                response. \textbf{Gate-to-gate wires are excluded.}
\pause
                \item   There is a space issue with the single gate design. The attenuation for the
                reflected signal is too small. The way to solve this problem is to change the layout
                to bring more open space for Ronen.
\pause
                \item   One good way to solve the signal attenuation is to utilize the open space for 
                antenna design. My idea is to fill the antenna with the rest of the circuit. Ronen
                agrees that it will bring the reflected back signal much stronger. 
\pause
                \item   Ronen needs a month after I handling him the standard cells. So that will push 
                my work's deadline to the end of Oct.
            \end{itemize}
    \end{itemize}
\end{frame}

\begin{frame}{Third Part}
    \begin{itemize}
        \item  \textit{CAD tools}
            \begin{itemize}
\pause
                \item   I can see the layout, but the bindkeys have not been set. So I can not change
                anything. Once I get that done, I will contact Ronen for single gate simulation.
\pause
                \item   130nm Design rules have not been found for Ronen's design. I will also get
                this part as fast as possible.
\pause          
                \item   Jamie can not help and I can not get help from Cadence. So his help from Cadence
                will be lagged for a long time.

            \end{itemize}
    \end{itemize}
\end{frame}

\subsection{Conclusion}
\begin{frame}
    \begin{itemize}
\pause          
        \item \textbf{CAD tools} Currently, we will working on the Nangate technology instead of 130nm.
                Current known ways to do the power simulation is to use encounter to simulate the vcd
                file from modelsim, which is not accurate. The accurate simulation need extraction from
                layout.
\pause          
        \item \textbf{Testbench} Testbench can be downloaded from Trusthub. We need to find of a testbench
                that makes sense in terms of area, leakage power and layout.
    \end{itemize}

\end{frame}

\begin{frame}{Time Stamp}
    Oct 15 , 2014
\end{frame}

\section{Layout of Hardware Trojan Circuits}
\subsection{Specs for Hidding HT}

\begin{frame}{A \textit{'Good Trojan'} in Hardware}
    \begin{itemize}
\pause          
        \item \textit{Size} The area of added circuit must be small due to extra energy consumption and 
                die size.
\pause          
        \item \textit{Side-Channel Leakage} In order leak information, the circuit must consume more energy
                for secret info transmitting. A \textit{'good Trojan'} must consume energy as small as
                possible but leak information as much as possible.
\pause          
        \item \textit{Trigger} The attacker must be hard to be triggered. This is due to extensive functional 
                testing.
    \end{itemize}
\end{frame}
   
\subsection{HT insertion AES-T100}
\begin{frame} {AES-T100 Explanation}

    \begin{itemize}
\pause          
        \item \textit{Side Channel Attack} This method is implemented with power side channel attack. It
                leaks the secret information through CDMA channel.
\pause          
        \item \textit{CDMA Leakage} The attacking circuit consists of a PRNG for spread spectrum. In
                this way, the leakage information will be distributed through many cycles so that the 
                leakage power analysis can not detect the information
\pause          
        \item \textit{Demo from Paper} It is implemented with FPGA. The author used 8 flipflops to mimic 
                a huge capacitor.
    \end{itemize}

\end{frame}

\begin{frame} {AES-T100}
    \begin{itemize}
\pause          
        \item \textbf{Functionality} Leak the keys of AES.
\pause          
        \item \textbf{Trigger Condition} Always on.
\pause          
        \item \textbf{Location} Highlighted.
    \end{itemize}
\end{frame}
 
\begin{frame} {Trojan Free Layout}
    \begin{figure}[p]
        \includegraphics[width=2.5in]{../img/AES_T100_TjFree.PNG}
        \caption{AES-T100 Trojan Free Circuit}
        \label{AES_T100_TjFree}
    \end{figure}
\end{frame}

\begin{frame} {Trojan Free Layout Without Filler Cells}
    \begin{figure}[p]
        \includegraphics[width=2.5in]{../img/AES_T100_TjFree_wnFillerCell.PNG}
        \caption{AES-T100 Trojan Free Circuit without Filler Cells}
        \label{AES_T100_TjFree_wnFillerCell}
    \end{figure}
\end{frame}
 
\begin{frame} {Trojan In Layout}
    \begin{figure}[p]
        \includegraphics[width=2.5in]{../img/AES_T100_TjIn.PNG}
        \caption{AES-T100 Trojan Inserted Circuit}
        \label{AES_T100_TjIn}
    \end{figure}
\end{frame}

\begin{frame} {Trojan Free Layout Without Filler Cells}
    \begin{figure}[p]
        \includegraphics[width=2.5in]{../img/AES_T100_TjIn_wnFillerCell.PNG}
        \caption{AES-T100 Trojan Inserted Circuit without Filler Cells}
        \label{AES_T100_TjIn_wnFillerCell}
    \end{figure}
\end{frame}
 
\subsection{Dummy Trojan}
\begin{frame} {b15-T100}
    \begin{itemize}
\pause          
        \item \textbf{Functionality} Slows down part of the cicuit by reducing the clock frequency by half.
\pause          
        \item \textbf{Trigger Condition} Observe 0xFF for the address bus for bits 8-15.
\pause          
        \item \textbf{Location} Tightly placed at the bottom left section of the layout.
\pause          
        \item \textbf{$P\&R$ Problem} I did not finish place and route.
    \end{itemize}
\end{frame}
 
\end{document}


