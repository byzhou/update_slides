% Copyright 2007 by Till Tantau
%
% This file may be distributed and/or modified
%
% 1. under the LaTeX Project Public License and/or
% 2. under the GNU Public License.
%
% See the file doc/licenses/LICENSE for more details.



\documentclass{beamer}

%
% DO NOT USE THIS FILE AS A TEMPLATE FOR YOUR OWN TALKS�!!
%
% Use a file in the directory solutions instead.
% They are much better suited.
%


% Setup appearance:

\usetheme{Darmstadt}
%\usetheme{CambridgeUS}
%\usetheme{Berkeley}
%\usetheme{Hannover}
%\usefonttheme[onlylarge]{structurebold}
\usefonttheme{professionalfonts}
%\usefonttheme{default}
\setbeamerfont*{frametitle}{size=\normalsize,series=\bfseries}
\setbeamertemplate{navigation symbols}{}
\setbeamercovered{transparent=30}
\setbeamertemplate{footline}{
    
    \leavevmode%
    \hbox{%
    \begin{beamercolorbox}[wd=.4\paperwidth,ht=2.25ex,dp=1ex,center]{author in head/foot}%
        \usebeamerfont{author in head/foot}\insertshortauthor
    \end{beamercolorbox}%

    \begin{beamercolorbox}[wd=.6\paperwidth,ht=2.25ex,dp=1ex,center]{title in head/foot}%
        \usebeamerfont{title in head/foot}\insertshorttitle\hspace*{3em}
        \insertframenumber{} / \inserttotalframenumber\hspace*{1ex}
    \end{beamercolorbox}}%
    \vskip0pt% 
    
    
}

% Standard packages

\usepackage[english]{babel}
\usepackage[latin1]{inputenc}
\usepackage{times}
\usepackage[T1]{fontenc}
\usepackage{libertine}
\usepackage{graphicx}
%\usepackage{natbib}
\usepackage{enumerate}
\usepackage{biblatex}


\usepackage{epstopdf}
\usepackage{tabularx}



% Setup TikZ

\usepackage{tikz}
\usetikzlibrary{arrows}
\tikzstyle{block}=[draw opacity=0.7,line width=1.4cm]
\addbibresource{updates.bib}
\let\oldfootnotesize\footnotesize
\renewcommand*{\footnotesize}{\oldfootnotesize\tiny}
%change the font here
\sc

% Author, Title, etc.

\title[ 2015 fall updates ] 
{%
    Hardware Security Project
  %
}

\author[Boyou Zhou]
{
  \textit{Boyou Zhou\inst{*}}
}

\institute[Boston University, MA]
{
  \inst{*}
  Boston Univeristy, MA
}

\date[fall 2014]
{Created on Feb 8 2014, Modified on \today}

% The main document

\begin{document}

\begin{frame}
  \titlepage
\end{frame}

\begin{frame}{Outline}
  \tableofcontents
\end{frame}

\AtBeginSubsection{
    \frame<beamer>{ 
    \frametitle{Outline}   
    \tableofcontents[currentsection,currentsubsection]
    }
}

\section{Plans for the ICCAD}
\begin{frame}{Basic Thoughts}
    \begin{itemize}
\pause
        \item [*] We can have the simulation including all the metal layers,
                material inside of standard cells, like poly-silicon.
\pause
        \item [*] Design a metal structure inside of metal layer, in
                \textit{Metal1} or other metal layers in order to improve the
                ability to detect Hardware Trojans.
    \end{itemize}
\end{frame}

\subsection{More Detailed Simulation}
\begin{frame}{Two possible methods to simulate the entire chip}
    \begin{itemize}
\pause
        \item [opt 1] First, we use the rectlinear decomposition to divide all the
            parts inside standard cells into rectangles. And then use the def
            file to locate all the cells positions. At last, we combine all
            these informations with metal connections to one file. The problem
            is that lef only contains {\color{red}\textbf{signal pins, power and
            ground pins, vias, and power and ground stripes}}.
\pause
        \item [opt 2] Use python to read the gds file and then use rectlinear
            decomposition to all the parts inside the chip.
    \end{itemize}
\end{frame}

\begin{frame}{Simple Idea on Rectlinear Decomposition}
\pause
    The basic algorithm has been applied to the \textit{Metal1} and it seems to
    be working. The next step is to ensure the accurate of the algorithm. The
    first order of testing the algorithm is to use the area to verify the
    program.
\end{frame}

\subsection{Metal Pattern}
\begin{frame}
\pause
    I need to look into the document to look for the specification of filler
    cell positions.
\end{frame}

\section{Reviews from DAC}
\begin{frame}
    Feb 17th, 2015
\end{frame}

\begin{frame}{The first reviewer, 2}
    \begin{itemize}
    \item 16 did such imaging for temperature measurements and Trojan
    detection. The drawback here and in 16 is that the imaging must be
    performed on unpackaged die. In BISA technique (HOST 13 and TCAD 14), they
    do not need unpacked ICs. 
\pause
    \item 
    If Trojans insertion is performed by replacing standard cells in the design
    with smaller custom cells, this will not impact the filler cells or
    watermark so such Trojans are undetectable.
\pause
    \item 
    Another limitation is the speed of the technique. ?The authors claim a few
    hours would be required to test an IC and they think this is an advantage.
\pause
    \item 
    The authors use FDTD simulation for most experiments, but do not describe
    the software or its limitations.
\pause
    \item
    Detection rates vs. SNR are given to show the accuracy of the approach, but
    the authors never mention what an expected SNR might be so the results are
    not as easy to interpret.
    \end{itemize}
\end{frame}

\begin{frame}{The second reviewer, 4}
    \begin{itemize}
        \item It would be better if there are details regarding design overhead
            and evaluation results using fabricated chips.
\pause
        \item How to engineer and where to insert the fill cells?
\pause
        \item Any rules to choose the location and number of embedded cells?
\pause
        \item Adding maximal metal into the cell which doesn't violate the design
            rules doesn't necessarily lead to no violations for the entire circuit.
\pause
        \item Inserting fill cells into blank area is easy, but should be careful
            if among certain functional cells, because it may interrupt the metal
            connections among functional cells. 
\pause
        \item Too many fill cells with high metal density in a certain area may
            cause polishing problem during fabrication, not high enough metal
            density may not create significant reflectance compare to its adjacent
            area in real chip.
    \end{itemize}
\end{frame}

\begin{frame}{The third reviewer, 4}
    \begin{itemize}
        \item The big issue is that the work appears to be theoretical with no
        real experimental data to validate these claims. The image quality and
        the speed may change when applied to real designs. 
    \end{itemize}
\end{frame}

\begin{frame}{The fourth reviewer, 2}
    \begin{itemize}
        \item Whether there is any process variation to this approach and
        how much. Second, a design to be checked for Trojan must be compared
        with a golden design for the pattern (watermark). How to obtain this
        golden design? 
\pause
        \item The 2$\%$ leakage overhead seems to be small, but the 0.1$\%$
        of the total area is still large for some Trojan. (for a chip with 2
        million gates, this 0.1$\%$ means 2000 gates!) It will be interesting and
        might be challenging to reduce this 0.1$\%$ to a much smaller number, maybe
        $10^{-6}$ or smaller. 
    \end{itemize}
\end{frame}

\begin{frame}{The fifth reviewer, 4}
    \begin{itemize}
        \item The limitations and/or comparison to existing like techniques are
        not exactly described. For example, I am not sure how much better
        detection this paper achieves compared to the imaging references below.

        while the paper cites a number of marginally related work, e.g.,
        $[2][13][15]$, the authors miss the important related work on using
        the optical and thermal imaging to detect hardware Trojans, in
        particular:
\pause
        \begin{itemize}
            \item {\color{red}P. Song et al., "MARVEL - Malicious Alteration Recognition and
            Verification by Emission of Light" HOST 2011}
\pause
            \item F. Stellari et al., "Functional block extraction for hardware security
            detection using time-integrated and time-resolved emission
            measurements," VTS 2014?
\pause
            \item Nowroz et al, "Novel Techniques for High-Sensitivity Hardware Trojan
            Detection Using Thermal and Power Maps", TCAD 2014
        \end{itemize}
    \end{itemize}
\end{frame}

\begin{frame}{The sixth reviewer, 3}
    \begin{itemize}
        \item By analyzing multiple optical responses of Trojan free chips, can
        the attacker identify the location of the fill cells? 
\pause
        \item I can?t see how the author can differentiate between figure 
        2.b and 3.b by looking at their images only as he/she claims.
\pause
        \item questionable claim of capability to deal with process variations
\pause
        \item need comparison against TJ Watson's PICO technique
    \end{itemize}
\end{frame}

\begin{frame}{The seventh reviewer, 5}
    \begin{itemize}
        \item Will modifying the fill cells affect the analog characteristics of
         the chip?
\pause
        \item For Replacement\_Type1 and Replacement\_Type2 tests, how many
         gates/cells were changed? What is the lower limit on the number of fill
         cells/gates that can be modified and still be detected?
\pause
        \item For Figure 4, how are these detection error rates calculated? Is
         this empirical or can you guarantee no false positives/negatives at a
         threshold of 0.65?
    \end{itemize}
\end{frame}

\section{python imported files}
\subsection{scripts problems}
\begin{frame}{info in gds}
    \begin{itemize}
        \item [*] gds file does not contain info of metal structures inside a
            standard cell. It contains the positions of the cells, types of the
            cells, and all the vias.
        \item [*] The info inside gds file is listed below.
        \\
        {\small
        \begin{itemize}
            \item [ ] CellReference("XNOR2\_X1", (26.98, 23.1), 0.0, None, True),
            \item [ ] CellReference("XNOR2\_X1", (22.8, 20.3), 180.0, None, False),
            \item [ ] CellReference("XNOR2\_X1", (20.33, 17.5), 180.0, None, True),
            \item [ ] CellReference("XNOR2\_X1", (22.42, 25.9), 0.0, None, True),
            \item [ ] CellReference("XNOR2\_X1", (21.47, 25.9), 180.0, None, True),
            \item [ ] CellReference("XNOR2\_X1", (21.47, 17.5), 0.0, None, False),
            \item [ ] CellReference("XNOR2\_X1", (28.31, 20.3), 0.0, None, False),
            \item [ ] CellReference("top\_VIA2", (11.385, 16.1), None, None, False),
            \item [ ] CellReference("top\_VIA8", (11.385, 16.1), None, None, False),
            \item [ ] CellReference("top\_VIA9", (11.385, 16.1), None, None, False),
        \end{itemize}
        }
    \end{itemize}
\end{frame}

\subsection{Standard cells in layout}
\begin{frame}{info from encounter}
    \begin{figure}[p]
        \includegraphics[width=2in]{../img/metal1stdcells.PNG}
        \caption{AES from DAC paper}
    \end{figure}
\end{frame}

\begin{frame}{info from encounter}
    \begin{figure}[p]
        \includegraphics[width=2in]{../img/simpleexample.PNG}~
        \includegraphics[width=2in]{../img/simpleexample2.PNG}~
        \caption{A much simpler example \footfullcite{wei2012hardware}}
    \end{figure}
\end{frame}

\section{Prep for ICCAD}
\subsection{Previous work}
\begin{frame}{IC counterfeiting defs}
    \begin{itemize}
        \item IC counterfeiting category \footfullcite{guin2014counterfeit}
        \begin{itemize}
            \item {\color{red}unauthorized copy}
            \item not conform to original design, model and/or performance
            standards
            \item off specs, defective or used design sold as new
            \item incorrect or false markings and/or documents
        \end{itemize}
    \end{itemize}
\end{frame}

\begin{frame}{FPGA IP protection}
    Main methods for FPGA IP protection.\footfullcite{zhang2013fpga}
    \begin{itemize}
        \item {\color{red}encryption} Commercially available encryption-based
        techniques are limited to single large FPGA configuration.
        \footfullcite{altera2009}
        \item {\color{red}encryption-based} licensing Requires
        TTY\footfullcite{maes2012pay}
        \item {\color{red}HW-IP binding methods} use mechanisms in scured ROM or
        flash memory. They are vulnerable to side-channel
        attacks.\footfullcite{alkabani2007remote}
        \item PUF\footfullcite{zhang2013fpga}
    \end{itemize}
\end{frame}

\begin{frame}{PUF with FSM}
    \begin{figure}[p]
        \includegraphics[width=4in]{../img/PUFFSM.PNG}
        \caption{PUF with FSM lock\footfullcite{zhang2013fpga}}
        \label{PUFinFSM}
    \end{figure}
\end{frame}

\begin{frame}{ASIC IP protection}
    Main methods for FPGA IP protection.\footfullcite{maes2012pay}
    \begin{itemize}
        \item TTY, encryption related
            \footfullcite{alkabani2007remote}
            \footfullcite{roy2008epic}
            \footfullcite{roy2008protecting}
            \footfullcite{maes2009analysis}
        \item PUF
    \end{itemize}
\end{frame}

\subsection{Optical PUF}
\begin{frame}{optical PUF against illegal copy of IP}
    \begin{figure}[p]
        \includegraphics[width=2in]{../img/metastability.PNG}~
        \includegraphics[width=2in]{../img/laserstimulation.PNG}
        \caption{metastability\footfullcite{wieczorek2014dual} and laser
        stimulation \footfullcite{nedospasov2013invasive}}
    \end{figure}
\end{frame}

\begin{frame}{optical PUF}
    This method requires IP core provide layouts.
    \begin{itemize}
\pause
        \item Design LFSR and seed generator inside IP core. 
\pause
        \item With the scan with laser, metastability will not be maintained.
        Therefore, the seeds will be generated. And so does the key. 
\pause
        \item For different users, LFSR generates different keys.
    \end{itemize}
\end{frame}

\end{document}
