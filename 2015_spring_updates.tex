% Copyright 2007 by Till Tantau
%
% This file may be distributed and/or modified
%
% 1. under the LaTeX Project Public License and/or
% 2. under the GNU Public License.
%
% See the file doc/licenses/LICENSE for more details.



\documentclass{beamer}

%
% DO NOT USE THIS FILE AS A TEMPLATE FOR YOUR OWN TALKS�!!
%
% Use a file in the directory solutions instead.
% They are much better suited.
%


% Setup appearance:

\usetheme{Darmstadt}
%\usetheme{CambridgeUS}
%\usetheme{Berkeley}
%\usetheme{Hannover}
%\usefonttheme[onlylarge]{structurebold}
\usefonttheme{professionalfonts}
%\usefonttheme{default}
\setbeamerfont*{frametitle}{size=\normalsize,series=\bfseries}
\setbeamertemplate{navigation symbols}{}
\setbeamercovered{transparent=30}
\setbeamertemplate{footline}{
    
    \leavevmode%
    \hbox{%
    \begin{beamercolorbox}[wd=.4\paperwidth,ht=2.25ex,dp=1ex,center]{author in head/foot}%
        \usebeamerfont{author in head/foot}\insertshortauthor
    \end{beamercolorbox}%

    \begin{beamercolorbox}[wd=.6\paperwidth,ht=2.25ex,dp=1ex,center]{title in head/foot}%
        \usebeamerfont{title in head/foot}\insertshorttitle\hspace*{3em}
        \insertframenumber{} / \inserttotalframenumber\hspace*{1ex}
    \end{beamercolorbox}}%
    \vskip0pt% 
    
    
}

% Standard packages

\usepackage[english]{babel}
\usepackage[latin1]{inputenc}
\usepackage{times}
\usepackage[T1]{fontenc}
\usepackage{libertine}
\usepackage{graphicx}
%\usepackage{natbib}
\usepackage{enumerate}


\usepackage{epstopdf}
\usepackage{tabularx}



% Setup TikZ

\usepackage{tikz}
\usetikzlibrary{arrows}
\tikzstyle{block}=[draw opacity=0.7,line width=1.4cm]

%change the font here
\sc

% Author, Title, etc.

\title[ 2015 fall updates ] 
{%
    Hardware Security Project
  %
}

\author[Boyou Zhou]
{
  \textit{Boyou Zhou\inst{*}}
}

\institute[Boston University, MA]
{
  \inst{*}
  Boston Univeristy, MA
}

\date[fall 2014]
{Created on Feb 8 2014, Modified on \today}

% The main document

\begin{document}

\begin{frame}
  \titlepage
\end{frame}

\begin{frame}{Outline}
  \tableofcontents
\end{frame}

\AtBeginSubsection{
    \frame<beamer>{ 
    \frametitle{Outline}   
    \tableofcontents[currentsection,currentsubsection]
    }
}

\section{Plans for the ICCAD}
\begin{frame}{Basic Thoughts}
    \begin{itemize}
\pause
        \item [*] We can have the simulation including all the metal layers,
                material inside of standard cells, like poly-silicon.
\pause
        \item [*] Design a metal structure inside of metal layer, in
                \textit{Metal1} or other metal layers in order to improve the
                ability to detect Hardware Trojans.
    \end{itemize}
\end{frame}

\subsection{More Detailed Simulation}
\begin{frame}{Two possible methods to simulate the entire chip}
    \begin{itemize}
\pause
        \item [*] First, we use the rectlinear decomposition to divide all the
            parts inside standard cells into rectangles. And then use the def
            file to locate all the cells positions. At last, we combine all
            these informations with metal connections to one file.
\pause
        \item [*] Use python to read the gds file and then use rectlinear
            decomposition to all the parts inside the chip.
    \end{itemize}
\end{frame}

\begin{frame}{Simple Idea on Rectlinear Decomposition}
\pause
    The basic algorithm has been applied to the \textit{Metal1} and it seems to
    be working. The next step is to ensure the accurate of the algorithm. The
    first order of testing the algorithm is to use the area to verify the
    program.
\end{frame}

\subsection{Metal Pattern}
\begin{frame}
\pause
    I need to look into the document to look for the specification of filler
    cell positions.
\end{frame}

\end{document}
